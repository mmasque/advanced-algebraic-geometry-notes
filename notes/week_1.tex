\chapter{Grothendieck's theorem, semistability of curves}
The lecturer this week was Yagna Dutta.
We skip the motivation for the course.
We skip the definitions of presheaves and sheaves.
\section{Grothendieck's theorem}
\begin{definition}
	Let $X$ be a scheme. A sheaf $\sh{F}$ of $\OO_X$-modules is \emph{free of finite rank} $r$ if there exists $r \in \ZZ_{\geq 0}$ such that $\sh{F} \cong \OO_x^{\oplus r}$. If there exists a cover $\{U_i\}$ of $X$ such that $\sh{F}|_{U_i} \cong \OO_{U_i}^{\oplus r}$ we say that $\sh{F}$ is \emph{locally free} of rank $r$.
\end{definition}

\begin{definition}
	Let $X$ be a scheme. A \emph{vector bundle} of rank $r$ over $X$ is a scheme $E$ and a morphism $\pi: E \to X$ together with additional data consisting of an open covering of $\{U_\alpha\}$ of $X$ 
	and isomorphisms $\psi_\alpha: \pi^{-1}(U_\alpha) \to \Aa^r_{U_\alpha}$, such that for any $\alpha, \beta$ and any open affine subset $V = \Spec A \subset U_\alpha \cap U_\beta$ the automorphism $\psi_{\alpha\beta} = \psi_\alpha \circ \psi^{-1}_\beta$ of $\Aa^r_{V} = \Spec A [x_1, \ldots, x_r]$ is \emph{linear}. A vector bundle of rank 1 is called a \emph{line bundle}.
\end{definition}

\begin{rmk}
	There is a one-to-one correspondence between isomorphism classes of locally free sheaves of rank $r$ on a scheme $X$ and isomorphism classes of vector bundles of rank $r$ over $X$ (see \cite{hartshorne2013algebraic} Exercise II.5.18).
\end{rmk}

\begin{example}
	Let $X = \Spec(k)$. Given a rank $r$ there is a unique vector bundle given by the tilde module on $k^{\oplus r}$ (which, since $X$ is a single point, is just the module itself).  
\end{example}

\begin{example}
	Let $X = \PP^1$. Then $\OO_X(d)$ is a line bundle for all $d \in \ZZ$. Why? Look at the appendix for a definition of $\OO_X$-module. On the standard opens we get ${R_i \cdot X_i^d}$, and $\OO(U_i) = R_i$ by definition of projective space (look at the appendix), so $\OO_X(d)(U_i)$ is a free module of rank 1 over $\OO_X(U_i)$. 
\end{example}

\begin{theorem}[Grothendieck's Theorem]
	Any vector bundle on $\PP^1$ can be written uniquely up to isomorphism as a direct sum of line bundles. 
\end{theorem}

We do not write down a proof here, nor did we see it in the lecture. Derek Sorensen gives a proof \cite{sorensen_classication_nodate} which I haven't read.

% \section{Classification of vector bundles on $\PP^1$}
% Skipped.

% \section{Semistability of curves}
% Skipped.

\section{Degree of a vector bundle on a curve}
We follow II.6 from \cite{hartshorne2013algebraic}.

\begin{definition}
	Let $k$ be an algebraically closed field. A \emph{curve} over $k$ is an integral separated scheme $X$ of finite type over $k$ of dimension 1 (this is topological dimension). If $X$ is proper over $k$, we say that $X$ is \emph{complete}. If all the local rings are regular then the curve is \emph{nonsingular}.
\end{definition}

Nonsingular curves are complete if and only if they are \emph{projective}.

\begin{definition}
	Let $X$ be a noetherian integral separated scheme which is regular (or nonsingular) in codimension one: every local ring of $X$ of dimension one is regular. A \emph{prime divisor} divisor on $X$ is a closed integral subscheme $Y$ of codimension one. A \emph{Weil divisor} is an element of the free group generated by the prime divisors. 
\end{definition}

Skipped: definition of principal divisor, divisor class.
\begin{definition}
	Let $D = \sum_i a_i Y_i$ be a Weil divisor on a curve $X$. We define the \emph{degree} of $D$ by \[
    	\deg(D) = \sum_i a_i.
    \]
\end{definition}


\begin{proposition}[Corollary 6.10 \cite{hartshorne2013algebraic}]
	A principal divisor on a complete nonsingular curve has degree zero.
\end{proposition}

This proposition tells us we have a well defined notion of degree on the divisor class. In particular, we can use this definition on the class of invertible sheaves on a curve.

\begin{definition}
	Let $X$ be a complete nonsingular curve. We define the degree of a line bundle $\sh{L} = \OO_X(D)$ by $
    	\deg(\sh{L}) = \deg(D).$ 
\end{definition}

So for a line bundle, we just look at the degree of the corresponding divisor class. We do not have such a correspondence for vector bundles. We are going to use the \emph{exterior algebra} of a sheaf to construct a canonical line bundle out of a vector bundle. In this way we can then define the degree of a vector bundle via its associated exterior algebra. Look at Exercise II.5.16 in \cite{hartshorne2013algebraic} for the construction.

\begin{definition}
	Let $\sh{F}$ be a locally free sheaf of rank $n$ on a complete nonsingular curve.
	The degree of $\sh{F}$ is defined as \[\deg(\sh{F}) = \deg(\bigwedge^n \sh{F}).\]
\end{definition}

This works because of the result given in Exercise II.5.16: for a rank $n$ locally free sheaf, the $n$th exterior power is an invertible sheaf.

\section{Semistability of curves}
\todo{ write.}
