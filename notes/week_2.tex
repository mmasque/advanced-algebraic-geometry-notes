\chapter{Some preliminary concepts}

\section{Ample line bundles}

\begin{definition}
	Let $X$ be a Noetherian scheme. An invertible sheaf $\mathcal{L}$ on $X$ is called \emph{ample} if for all coherent sheaves $\mathcal{F}$ on $X$, there exists $n_0$ (dependent on $\sh{F}$) such that for all $n \geq n_0$, $\sh{F} \otimes \sh{L}^n$ is generated by its global sections \footnote{This means that there exists a set of global sections $\{s_i\}$ such that for all $x \in X$, the corresponding germs generate the stalk over $\OO_{X,x}$.} (here $\sh{L}^n = \sh{L}^{\otimes n}$).
\end{definition}

A proposition to warm up.
\begin{proposition} Let $\sh{L}$ be an invertible sheaf on $X$ Noetherian.
	The following are equivalent: \begin{enumerate}
    	\item $\sh{L}$ is ample,
		\item $\sh{L}^m$ is ample for all $m > 0$,
		\item $\sh{L}^m$ is ample for some $m > 0$.
    \end{enumerate}
\end{proposition}

To prove $3 \implies 1$, show that $\sh{F} \otimes \sh{L}$ is coherent for any $\sh{F}$ coherent, and then use this to play a trick on indexes (Proposition II.7.5 \cite{hartshorne2013algebraic}).

\begin{definition}\label{def:very-ample}
	Let $X \to Y$ be a morphism of schemes. A sheaf $\sh{L}$ on a scheme $X$ is called \emph{very ample relative to $Y$} if there exists an immersion $i: X \to \PP^n_Y$ such that $\sh{L} \cong i^* \OO(1)$. Note that this makes $\sh{L}$ into an invertible sheaf, since pullbacks of invertible sheaves are invertible. \footnote{I am not sure if the immersion $i$ needs to be \emph{over $Y$}, that is, form a commutative diagram with the maps $X \to Y$ and $\PP^n_Y \to Y$.}. 
\end{definition}

The following theorem is II.7.6 in \cite{hartshorne2013algebraic}. The proof is about a page and a half there. 
\begin{theorem}\label{thm:ample-very-ample}
	Let $X$ be a scheme of finite type over a Noetherian ring $A$ (this means that $X$ is quasi-compact and $\Gamma(U, \OO_X)$ is a finitely generated $R$-algebra). Let $\sh{L}$ be an invertible sheaf on $X$. Then $\sh{L}$ is ample if and only if $\sh{L}^m$ is very ample over $\Spec A$ for some $m > 0$. 
\end{theorem}

\section{The Hilbert polynomial}

\begin{definition}
	Let $X$ be a projective scheme over a field $k$.  The \emph{Euler characteristic} of a coherent sheaf $E$ (Definition 7.6.3 in \cite{ag2}) is \[\chi(E) = \sum (-1)^i h^i(X, E),\] where $h^i(X, E) = \dim_k H^i(X, E)$, so the dimension of the ith cohomology group (which is a vector space over $k$, see Proposition 12.7.7 \cite{ag2}). Recall also that the main result of Algebraic Geometry 2 was that this is a finite dimensional vector space (\cite{ag2} Theorem 14.1.1).

\todo{Why is the above a finite sum?}
\end{definition}

\begin{definition}
	Let $X$ be a projective scheme over a field $k$. Recall that projective space is Noetherian, and that closed subschemes of Noetherian schemes are Noetherian (Exercise 7.6 \cite{ag2}).
	Fix an ample line bundle $\OO(1)$ in $X$. The \emph{Hilbert polynomial} of a coherent sheaf $E$ is \[
    	P(E): m \mapsto \chi(E \otimes \OO(m)).
    \]
\end{definition}



\section{Chow rings}
In this section, $X$ is an algebraic scheme. This means $X$ is a scheme over $k$ such that the map $X \to \Spec(k)$ is of finite type.
For this section, the lecturers used \cite{eisenbud20163264}. I also use the material there to fill in definitions, etc.

\begin{definition}
	The \emph{group of cycles} of $X$, denoted $Z(X)$ is the free abelian group generated by the set of reduced, irreducible subschemes of $X$ \footnote{\cite{eisenbud20163264} doesn't say whether these are closed, and the lecturer didn't seem sure when asked}. We get a dimension grading (at dimension $k$ we write $Z_k(X)$), and we recover the definition of \emph{divisor} by looking at  $Z_{n-1}(X)$ for a dimension $n$ scheme.
\end{definition}

\begin{definition}
	Let $X \subset Y$ be a closed subscheme. Let $Y_1, ..., Y_s$ be the isolated, reduced irreducible components of the support of $Y$ \footnote{\todo{ understand what this means}.}
	We define the \emph{cycle $Z$ associated to $Y$} to be the formal combination \[
    	\langle Y \rangle = \sum_i l_i Y_i
    \] where $l_i$ is the length of $\OO_{Y, Y_i}$, which is finite.
\end{definition}

Informally, we want an analogue of homotopy to relate two cycles. We will call it \emph{rational equivalence}. Two cycles will be rationally equivalent if there is a subvariety on $\PP^1 \times X$ whose restriction to two fibers $\{t_0\} \times X$ and $\{t_1\} \times X$ gives back the cycles. 

\begin{definition}
	Let $\operatorname{Rat}(X) \subset Z(X)$ be the subgroup generated by differences of the form 
	\[
    	\langle \Phi \cap (\{t_0\} \times X) \rangle - \langle \Phi \cap (\{t_1\} \times X) \rangle
    \] where $t_i \in \PP^1$ and $\Phi$ is a subvariety (reduced, irreducible subscheme) of $\PP^1 \times X$ not contained in any fiber $\{t\} \times X$. 
	The \emph{Chow group} of $X$ is the quotient $A(X) = Z(X) / \operatorname{Rat}(X)$.
\end{definition}
The Chow group is still graded by dimension, we take this for granted for now. In Chapter 1 of \cite{eisenbud20163264} there are details. 

\begin{notation}
It is convenient to define $A^k(X) = A_{\dim(X) - k}(X)$. 
\end{notation}

\begin{definition}
We want to make the Chow group into a ring. If $A, B$ are \emph{generically transverse} subvarieties of $X$, we define $[A] \circ [B] = [A \cap B]$.
\end{definition}

Given a morphism of schemes, we want associated maps between the Chow groups on the schemes: the pushforward and the pullback. We will not prove that these things are well defined.

\begin{definition}
	Let $f: X \to Y$ be a proper map of algebraic schemes. Let $A \subset X$ be a subvariety. 
	\begin{enumerate}
    	\item If $\dim f(A) < \dim A$ we define $f_*\langle A \rangle = 0$.
		\item $\dim f(A) = \dim A$ and $f|_A$ has degree $n$, then we define $f_*\langle A \rangle = n f(A)$. The degree $n$ is the degree of the field extension $K(A)/K(f(A))$, we blackbox this for now \footnote{\todo{ Look into this} p. 20 \cite{eisenbud20163264}}.
    \end{enumerate}
\end{definition}

\begin{definition}
	For a flat morphism $f: Y \to X$ we define $f^*(\langle A \rangle) = \langle f^{-1}(A) \rangle$.
\end{definition}

The last thing to define for Chow rings is an integral on dimension zero closed subschemes.

\begin{definition}
	Let $p: X \to \Spec(k)$ be a proper map, and let $\alpha \in A_0(X)$, so $\alpha$ is a class of zero-dimensional closed subschemes of $X$.
	We define the integral
	\[
    	\int_X \alpha := p_* \alpha.
    \] We can define the integral for $\alpha \in A(X)$ by just taking the zeroeth-dimensional component and computing its pushforward (the pushforward is a morphism of graded rings and $\Spec(k)$ is zero dimensional). Also, note that $A_0(\Spec k) \cong \ZZ$ \footnote{\todo{ why?}}. 
\end{definition}

\section{Grothendieck Riemann-Roch}

Now $X$ will denote a smooth projective variety over $k$ of characteristic $0$ with $k$ algebraically closed \footnote{Does this mean a \emph{nonsingular} projective variety over $k$ ...?}.

\begin{rmk}
	The class $A^1(X)$ is the rational equivalence class of reduced irreducible subschemes of $X$ of codimension $1$. This is the rational class of Weil divisors (one should square away the prerequisites for each definition). Furthermore, in the lecture it was claimed that this is actually the class group, i.e. that the subgroup $\operatorname{Rat}$ is actually the principal divisors.
\end{rmk}

\begin{theorem}
	There exists a unique way to assign to a vector bundle $\sh{E}$ on $X$ a class \[
    	c(\sh{E}) \in \bigoplus^\infty_{k=0} A^k(X),
    \] which we write $c(\sh{E}) = 1 + c_1(\sh{E}) + c_2(\sh{E}) + \dots$, such that the following properties are satisfied: 
	\begin{enumerate}
    	\item If $\sh{E} = \sh{L}$ is a line bundle then $c(\sh{L}) = 1 + c_1(\sh{L})$. 
		\item \todo{ I don't undertand this, skipped, see handwritten notes}.
		\item \emph{Whitney's formula}: given a short exact sequence \[
        	0 \to \sh{E} \to \sh{F} \to \sh{G} \to 0
        \] of vector bundles, we have $c(\sh{F}) = c(\sh{E}) \cdot c(\sh{G}) \in A(X)$. 
		\item If $\phi: Y \to X$ is a morphism of projective varieties then \[
        	\varphi^*(c(\sh{E})) = c(\varphi^*(\sh{E})) \footnote{although we only defined the pullback for some morphisms. In Theorem 1.16 \cite {eisenbud20163264} the pullback is defined for a morphism of projective varieties.}
        \]
    \end{enumerate} We call this the \emph{chern class} of $\sh{E}$.
\end{theorem}

\begin{corollary}
	Let $\sh{E}$ be a finite direct sum of line bundles: $\sh{E} = \sum^r \sh{L}_i$. Then the Chern class is the product \[
    	c(\sh{E}) = \prod c(\sh{L}_i) = \prod (1 + c_1(\sh{L_i})).
    \]
\end{corollary}

\begin{proof}
	Apply Whitney's formula successively.
\end{proof}

\begin{theorem}[Splitting principle]
	Any identity of chern classes that is true for a direct sum of line bundles is true in general.
\end{theorem}

\begin{definition}
	We call $c(T_X)$ the \emph{Chern class of $X$}, for $T_X$ the tangent bundle of $X$.
\end{definition}

\begin{definition}[Grothendieck group]
	We define the \emph{Grothendieck group} to be \[
    	K(X) = \ZZ[\text{isomorphism classses of vector bundles on X}]/\sim
    \] with direct sum giving the group operation, where $[A] \sim [B] + [C]$ if there is a short exact sequence $0 \to A \to B \to C \to 0$. We make this into a ring via tensor product.
\end{definition}

We would like a ring morphism $K(X) \to A(X)$. The natural candidate is the Chern class. But in general we see that $c(\sh{E} \otimes \sh{F}) \neq c(\sh{E})\cdot c(\sh{F})$. We introduce the \emph{Chern character} to fix this in a technical way.

\begin{definition}
	Let $\sh{E}$ be a vector bundle on $X$. From the splitting principle we can write $c(\sh{E}) = \prod^r (1 + \alpha_i)$, for $\alpha_i \in A_0(X)$. 
	We define the \emph{Chern character} of $\sh{E}$ by \[
    	\ch(\sh{E}) := \sum^r e^{\alpha_i}, 
    \] via the usual series definition of the exponential. Sometimes we write a subscript: $\ch_j(\sh{E}) := \sum^j e^{\alpha_i}$.
\end{definition}

\begin{example}~
	\begin{enumerate}
		\setcounter{enumi}{-1}
    	\item $\ch_0(\sh{E}) = \rk(\sh{E})$,
		\item $\ch_1(\sh{E}) = c_1(\sh{E})$,
		\item $\ch_2(\sh{E}) = (c_1(\sh{E})^2 - 2c_2(\sh{E}))/{2}$.
    \end{enumerate}
\end{example}

We will see later (I hope) how we can deal with this in the setting of coherent sheaves.

We need one more definition before Riemann-Roch.

\begin{definition}
	Let $\sh{E}$ be a vector bundle on $X$. Write $c(\sh{E}) = \prod (1+\alpha_i)$. 
	Define the \emph{Todd class} of $\sh{E}$ to be \[
    	\td(\sh{E}) = \prod^r_{i=1} \frac{\alpha_i}{1-e^{-\alpha_i}}.
    \]
\end{definition}

\begin{theorem}[Grothendieck Riemann-Roch]
	If $X$ is a smooth projective variety and $\sh{F}$ a coherent sheaf on $X$, then \[
    	\chi(\sh{F}) = \int_X \ch{\sh{F}} \cdot \operatorname{Td}(X)
    \] for $\operatorname{Td}(X) = \td(T_x)$.
	\newline
	For $\sh{F}_1, \sh{F}_2$ we have \[
    	\chi(\sh{F}_1, \sh{F}_2) = \sum (-1)^i \dim \operatorname{Ext}^1(\sh{F}_1, \sh{F}_2) = \int_X \ch(\sh{F}_1^\vee)\ch(\sh{F}_2) \td{X}.
    \]
\end{theorem}
