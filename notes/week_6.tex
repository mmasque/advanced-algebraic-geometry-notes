\chapter{More flatness, Castelnuovo-Mumford regularity}
The lecturer this week was Yagna Dutta.
\section{More flatness and Serre's theorems}
\begin{theorem}[{\cite[Theorem~III.9.9]{hartshorne2013algebraic}}]\label{thm:flat-hilbert-constant}
	Let $Y$ be a smooth projective variety. Let $\OO_Y(1)$ be a very ample line bundle. Let $S$ be a noetherian integral scheme. Let $X = Y\times_k S$ and let $f: X \to S$ be the projection map. Let $\sh{F}$ be a coherent sheaf on $X$. The following are equivalent 
	\begin{enumerate}[label=(\roman*)]
    	\item $\sh{F}$ is flat over $S$; 
		\item there exists an integer $m > 0$ such that $f_*(\sh{F}\otimes \OO_X(m))$ is locally free;
		\item The Hilbert polynomial $P(\sh{F}|_{X_s}, t)$ does not depend on $s \in S$. 
    \end{enumerate}
\end{theorem}
The proof is written out in full in Hartshorne for a case this theorem reduces to. Note that one can relax the assumption that $\OO_Y(1)$ is very ample to just ample, and that this will be needed in further lectures. It uses the following theorems by Serre. \todo{Look at the proof of 9.9 in Hartshorne, it looks pretty.}
\begin{theorem}
	Let $X$ be a projective scheme over a noetherian ring $A$, let $\OO(1)$ be a very ample invertible sheaf on $X$, and let $\sh{F}$ be a coherent $\OO_X$-module. Then there is an integer $n_0$ such that for all $n \geq n_0$ the sheaf $\sh{F}(n)$ can be generated by a finite number of global sections.
\end{theorem}
\begin{theorem}[{\cite[Theorem~III.5.2]{hartshorne2013algebraic}}]
	Let $X$ be a projective scheme over a noetherian ring $A$, and let $\OO_X(1)$ be a very ample invertible sheaf on $X$ over $\Spec A$. Let $\sh{F}$ be a coherent sheaf on $X$. Then \begin{enumerate}[label=(\alph*)]
    	\item For each $i \geq 0$, $H^i(X, \sh{F})$ is a finitely generated $A$-module;
		\item there is an integer $n_0$, depending on $\sh{F}$, such that for each $i > 0$ and each $n \geq n_0$, $H^i(X, \sh{F}(n)) = 0$.
    \end{enumerate}
\end{theorem}
\begin{proof}
	By \cref{def:very-ample} there is a closed immersion $i: X \to \PP^r_A$ of schemes over $A$ for some $r$ such that $\OO_X(1) = i^*\OO_{\PP^r}(1)$. If $\sh{F}$ is coherent on $X$ then $i_*\sh{F}$ is coherent on $\PP^r_A$ (\cite[Exercise II.5.5]{hartshorne2013algebraic}) and the cohomology is the same (\cite[Lemma~III.2.10]{hartshorne2013algebraic}, the idea is that flasque resolutions are preserved under pushforwards). So we reduce to $X = \PP^r_A$. In this setting the theorem holds for sheaves of the form $\OO_X(q)$ by explicit calculation (\cite[Theorem~II.5.1]{hartshorne2013algebraic}). For finite direct sums we thus also have this. We now use descending induction on $i$. For $i > r$ we have $H^i(X ,\sh{F}) = 0$. 
	In general, a coherent sheaf $\sh{F}$ can be put into a short exact sequence \[
    	0 \to \sh{R} \to \sh{E} \to \sh{F} \to 0
    \] where $\sh{R}$ is coherent and $\sh{E}$ is a finite direct sum of sheaves $\OO(q_i)$ by \cite[Corollary~II.5.18]{hartshorne2013algebraic}. We get the long exact sequence \[
    	\dots \to H^i(X, \sh{E}) \to H^i(X, \sh{F}) \to H^{i+1}(X, \sh{R}) \to \dots .
    \] The module on the left is finitely generated because we have the statement for finite direct sums of $\OO(q_i)$. The module on the right is finitely generated by the inductive hypothesis. Since these are modules over a noetherian ring they are both, noetherian, and thus is $H^i(X, \sh{F})$ is noetherian and so finitely generated. Similar but easier argument for the twist vanishing.
\end{proof}

\section{Castelnuovo-Mumford regularity}

\begin{definition}
	Let $X$ be a smooth projective variety with $\OO_X(1)$ an ample, basepoint-free (i.e. globally generated) $\OO_X$ module. Then a coherent sheaf $\sh{F}$ on $X$ is \emph{$m$-regular} for $m \in \ZZ$ if \[
    	H^i(X, \sh{F}(m-i)) = 0 
    \] for all $i > 0$. 
\end{definition}

\begin{theorem}
	Let $\sh{F}$ be an $m$-regular sheaf on $X$.
	\begin{enumerate}
    	\item $\sh{F}(m)$ is globally generated.
		\item For all $k > 0$ the map \[
        	H^0(X, \sh{F}(m)) \otimes H^0(\OO_X(k)) \to H^0(\sh{F}(m+k))
        \] is surjective.
		\item If $\sh{F}$ is $m$-regular it is also $m'$-regular for all $m' > m$.
    \end{enumerate}
\end{theorem}

The following theorem is only here because we will use it in the next example. But it is nice to know. 
\begin{theorem}[Kodaira Vanishing Theorem, {\cite[Remark~III.7.15]{hartshorne2013algebraic}}]
	Let $X$ be a projective nonsingular variety of dimension $n$ over $\CC$, and let $\sh{L}$ be an ample invertible sheaf on $X$. Then \begin{enumerate}
    	\item $H^i(X, \sh{L}\otimes \omega) = 0$ for $i > 0$;
		\item $H^i(X, \sh{L}^{-1}) = 0$ for $i < n$. 
    \end{enumerate}
\end{theorem}
It's also good to recall Grothendieck's vanishing theorem (one of?).
\begin{theorem}[Grothendieck {\cite[Theorem~2.7]{hartshorne2013algebraic}}]
	Let $X$ be a noetherian topological space of dimension $n$. Then for all $i > n$ and all sheaves of abelian groups $\sh{F}$ on $X$, we have $H^i(X, \sh{F}) = 0$. 	
\end{theorem}

\begin{example}
	For $X = \PP^n_\CC$ and $\sh{F} = \OO(l)$, 
	we are interested in the values of $m$ that make $\sh{F}$ $m$-regular. If $i > n$ then the cohomology vanishes. 
	If $i = n$ we have \[
    	H^n(X, \OO(l + m - n)) = H^0(X, \OO(-l-m + n -n-1))
    \]
	by Serre duality and that $\omega = \OO(-n-1)$ (see the reference in the proof of \cite[Theorem~III.7.1]{hartshorne2013algebraic}).
	This vanishes iff $m > -l-1$. Assuming this bound now, we claim 
	If $0 < i \leq n$ the cohomology also vanishes.
	Write $l + m - i = a - n - 1$ for $a = l+m-i+n+1 > 0$. 
	Then we have $H^i(X, \OO(l+m-i)) = H^i(\OO(a) \otimes \omega)$ and $\OO(a)$ is ample since $a > 0$, so we can apply Kodaira's Vanishing theorem.
\end{example}

\begin{theorem}
	Suppose $\sh{F}$ is a coherent sheaf on $X = \PP^n$, and suppose we have an injection \[
    	\sh{F} \hookrightarrow \OO_X^{\oplus p}.
    \] Then there exists a numerical polynomial (that's a map $\ZZ \to \ZZ$) $m_{p,n}(t_0, \dots, t_n)$ such that if $P(\sh{F}, r) = \sum^n_{i=0}a_i {r \choose i}$ for some $r \in \ZZ$ then $\sh{F}$ is $m_{p, n}(a_0, \dots, a_n)$-regular.
\end{theorem}

