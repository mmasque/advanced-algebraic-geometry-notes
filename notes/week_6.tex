\chapter{More flatness, Castelnuovo-Mumford regularity}
\section{More flatness and Serre's theorems}
\begin{theorem}[{\cite[Theorem~III.9.9]{hartshorne2013algebraic}}]
	Let $Y$ be a smooth projective variety. Let $\OO_Y(1)$ be a very ample line bundle. Let $S$ be a noetherian integral scheme. Let $X = Y\times_k S$ and let $f: X \to S$ be the projection map. Let $\sh{F}$ be a coherent sheaf on $X$. The following are equivalent 
	\begin{enumerate}[label=(\roman*)]
    	\item $\sh{F}$ is flat over $S$; 
		\item there exists an integer $m > 0$ such that $f_*(\sh{F}\otimes \OO_X(m))$ is locally free;
		\item The Hilbert polynomial $P(\sh{F}|_{X_s}, t)$ does not depend on $s \in S$. 
    \end{enumerate}
\end{theorem}
The proof is written out in full in Hartshorne. It uses the following theorems by Serre. \todo{Look at the proof of 9.9 in Hartshorne, it looks pretty.}
\begin{theorem}
	Let $X$ be a projective scheme over a noetherian ring $A$, let $\OO(1)$ be a very ample invertible sheaf on $X$, and let $\sh{F}$ be a coherent $\OO_X$-module. Then there is an integer $n_0$ such that for all $n \geq n_0$ the sheaf $\sh{F}(n)$ can be generated by a finite number of global sections.
\end{theorem}
\begin{theorem}[{\cite[Theorem~III.5.2]{hartshorne2013algebraic}}]
	Let $X$ be a projective scheme over a noetherian ring $A$, and let $\OO_X(1)$ be a very ample invertible sheaf on $X$ over $\Spec A$. Let $\sh{F}$ be a coherent sheaf on $X$. Then \begin{enumerate}[label=(\alph*)]
    	\item For each $i \geq 0$, $H^i(X, \sh{F})$ is a finitely generated $A$-module;
		\item there is an integer $n_0$, depending on $\sh{F}$, such that for each $i > 0$ and each $n \geq n_0$, $H^i(X, \sh{F}(n)) = 0$.
    \end{enumerate}
\end{theorem}
\begin{proof}
	By \cref{def:very-ample} there is a closed immersion $i: X \to \PP^r_A$ of schemes over $A$ for some $r$ such that $\OO_X(1) = i^*\OO_{\PP^r}(1)$. If $\sh{F}$ is coherent on $X$ then $i_*\sh{F}$ is coherent on $\PP^r_A$ (\cite[Exercise II.5.5]{hartshorne2013algebraic}) and the cohomology is the same (\cite[Lemma~III.2.10]{hartshorne2013algebraic}, the idea is that flasque resolutions are preserved under pushforwards). So we reduce to $X = \PP^r_A$. In this setting the theorem holds for sheaves of the form $\OO_X(q)$ by explicit calculation (\cite[Theorem~II.5.1]{hartshorne2013algebraic}). For finite direct sums we thus also have this. We now use descending induction on $i$. For $i > r$ we have $H^i(X ,\sh{F}) = 0$. 
	In general, a coherent sheaf $\sh{F}$ can be put into a short exact sequence \[
    	0 \to \sh{R} \to \sh{E} \to \sh{F} \to 0
    \] where $\sh{R}$ is coherent and $\sh{E}$ is a finite direct sum of sheaves $\OO(q_i)$ by \cite[Corollary~II.5.18]{hartshorne2013algebraic}. We get the long exact sequence \[
    	\dots \to H^i(X, \sh{E}) \to H^i(X, \sh{F}) \to H^{i+1}(X, \sh{R}) \to \dots .
    \] The module on the left is finitely generated because we have the statement for finite direct sums of $\OO(q_i)$. The module on the right is finitely generated by the inductive hypothesis. Since these are modules over a noetherian ring they are both, noetherian, and thus is $H^i(X, \sh{F})$ is noetherian and so finitely generated. Similar but easier argument for the twist vanishing.
\end{proof}

\section{Castelnuovo-Mumford regularity}
